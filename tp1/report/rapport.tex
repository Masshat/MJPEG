\documentclass[12pt]{article}

\usepackage[utf8]{inputenc}
\usepackage[T1]{fontenc}
\usepackage[french]{babel}
\usepackage{float}
\usepackage[top=3cm, bottom=3cm, left=3cm, right=3cm]{geometry}
\usepackage{graphicx}
\usepackage{array}

\floatplacement{figure}{H}
\newcommand{\HRule}{\rule{\linewidth}{0.5mm}}

\begin{document}
\begin{titlepage}
  \begin{center}
    \textsc{\LARGE Université Pierre et Marie Curie}\\[1.5cm]
    \includegraphics[height=1cm]{upmc.png}\\[1.5cm]
    \textsc{\Large Rapport Archi 4 }\\[2cm]
    \HRule \\[1cm]
    \textsc{\huge TP1 MJPEG }\\[0.5cm]
    \HRule \\[1cm]
    % Author and supervisor
    \noindent
    \begin{minipage}[t]{0.55\textwidth}
      \begin{flushleft} \large
        \emph{Auteurs:}\\
        BITAM \textsc{Massine}\\
        TOUMLILT \textsc{Ilyas}
      \end{flushleft}
    \end{minipage}%
    \begin{minipage}[t]{0.47\textwidth}
      \begin{flushright} \large
        \emph{Encadrant:} \\
        MEUNIER \textsc{quentin}
      \end{flushright}
    \end{minipage}
    \vfill
    % Bottom of the page
    {\large \today}
  \end{center}
\end{titlepage}

\section*{Question 1}

\textbf{->}Fichiers présents avant la commande :\\
\$ ls *

splitmsg.py

src:\\
consumer.c  consumer.task  producer.c  producer.task\\

\textbf{->}Fichiers crées après la commande :\\
\$ ls *

exe.posix\\

posix:\\
barrier.deps      desc.o            lock.deps  mwmr.deps\\
barrier.o         dsx\_hw\_init.c     lock.o     mwmr.o\\
consumer.deps     dsx\_hw\_init.deps  log.deps   producer.deps\\
consumer.o        dsx\_hw\_init.o     log.o      producer.o\\
consumer\_proto.h  endianness.deps   main.deps  producer\_proto.h\\
desc.c            endianness.o      main.o\\
desc.deps         exe.posix         Makefile\\

\section*{Question 2}
Producteur produit ``...World'' et affiche ``Hello...'', le consommateur cosomme le ``...World'' et l'affiche.

\section*{Question 3}
Dans la première partie, on définit les tâches de notre application et les interconnexions entre elles (Les fifo de communication), c'est la construction du TCG. Et dans la seconde partie, on génère le code pour l'exécution sur la plateforme matérielle souhaitée.

\section*{Question 4}
Ce qui n'est pas présent dans la description DSX/L, et qui est indispensable 
à la construction et à l'exécution de l'application, c'est la description
des tâches.

\section*{Question 5}
Nous observons que l'image source ( plan avion ) fait des rotations, la 
rapidité de la rotation laisse croire que c'est une croix.
On remarque également que la vitesse de rotation est variable, en effet ceci 
est certainement du au fait que certaines tâches prennent plus de temps de 
traitement que d'autres.

\end{document}
