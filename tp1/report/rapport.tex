\documentclass[12pt]{article}

\usepackage[utf8]{inputenc}
\usepackage[T1]{fontenc}
\usepackage[french]{babel}
\usepackage{float}
\usepackage[top=3cm, bottom=3cm, left=3cm, right=3cm]{geometry}
\usepackage{graphicx}
\usepackage{array}

\floatplacement{figure}{H}
\newcommand{\HRule}{\rule{\linewidth}{0.5mm}}

\begin{document}
\begin{titlepage}
  \begin{center}
    \textsc{\LARGE Université Pierre et Marie Curie}\\[1.5cm]
    \includegraphics[height=1cm]{upmc.png}\\[1.5cm]
    \textsc{\Large Rapport Archi 4 }\\[2cm]
    \HRule \\[1cm]
    \textsc{\huge TP1 MJPEG }\\[0.5cm]
    \HRule \\[1cm]
    % Author and supervisor
    \noindent
    \begin{minipage}[t]{0.55\textwidth}
      \begin{flushleft} \large
        \emph{Auteurs:}\\
        BITAM \textsc{Massine}\\
        TOUTMLILT \textsc{Ilyas}
      \end{flushleft}
    \end{minipage}%
    \begin{minipage}[t]{0.47\textwidth}
      \begin{flushright} \large
        \emph{Encadrant:} \\
        MEUNIER \textsc{quentin}
      \end{flushright}
    \end{minipage}
    \vfill
    % Bottom of the page
    {\large \today}
  \end{center}
\end{titlepage}

\section*{Question 1}

\$ ls *

splitmsg.py

src:
consumer.c  consumer.task  producer.c  producer.task

\$ ls *

exe.posix

posix:
barrier.deps      desc.o            lock.deps  mwmr.deps
barrier.o         dsx\_hw\_init.c     lock.o     mwmr.o
consumer.deps     dsx\_hw\_init.deps  log.deps   producer.deps
consumer.o        dsx\_hw\_init.o     log.o      producer.o
consumer\_proto.h  endianness.deps   main.deps  producer\_proto.h
desc.c            endianness.o      main.o
desc.deps         exe.posix         Makefile

\section*{Question 2}
Producteur produit ...World et affiche Hello..., le consommateur cosomme le ...World et l'affiche.

\section*{Question 3}
Une première partie, où l'on défini les tâches de notre application et les interconnexions entre elles(Les fifo de communication), C'est la construction du TCG. Et une seconde partie dans laquelle, on génère le code pour l'éxecution sur la plateforme matérielle souhaitée.
\section*{Question 4}

\section*{Question 5}
nous observons une image qui fait des rotation, la vitesse est trop rapide, mais nous avons vu que le fichier plan est l'image d'un avion,
vu que la vitesse est trop rapide il apparait que le sens de rotation change de temps en temps.

\end{document}
