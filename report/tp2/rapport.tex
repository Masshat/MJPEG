\documentclass[12pt]{article}

\usepackage[utf8]{inputenc}
\usepackage[T1]{fontenc}
\usepackage[french]{babel}
\usepackage{float}
\usepackage[top=3cm, bottom=3cm, left=3cm, right=3cm]{geometry}
\usepackage{graphicx}
\usepackage{array}

\floatplacement{figure}{H}
\newcommand{\HRule}{\rule{\linewidth}{0.5mm}}

\begin{document}
\begin{titlepage}
  \begin{center}
    \textsc{\LARGE Université Pierre et Marie Curie}\\[1.5cm]
    \includegraphics[height=1cm]{upmc.png}\\[1.5cm]
    \textsc{\Large Rapport Archi 4 }\\[2cm]
    \HRule \\[1cm]
    \textsc{\huge TP2 MJPEG }\\[0.5cm]
    \HRule \\[1cm]
    % Author and supervisor
    \noindent
    \begin{minipage}[t]{0.55\textwidth}
      \begin{flushleft} \large
        \emph{Auteurs:}\\
        BITAM \textsc{Massine}\\
        TOUMLILT \textsc{Ilyas}
      \end{flushleft}
    \end{minipage}%
    \begin{minipage}[t]{0.47\textwidth}
      \begin{flushright} \large
        \emph{Encadrant:} \\
        MEUNIER \textsc{quentin}
      \end{flushright}
    \end{minipage}
    \vfill
    % Bottom of the page
    {\large \today}
  \end{center}
\end{titlepage}

\section*{Question 1}
La syntaxe utilisée par DSX pour exprimer que le port P0 du composant matériel C0 est connecté au 
port P1 du composant matériel C1, est la division entière //

\section*{Question 2}
Objets logiciels à placer dans une tâche :\\
- run : ...\\
- stack\\
- desk \\
dans le processeur :\\
- private :\\
- shared \\
dans un canal mwmr :\\
- buffer\\
- status\\

\section*{Question 3}
On observe, comme pour le TP précédent, les affichages ``Producer: Hello...'' et ``Consumer: ...World'', la simulation sur le modèle systemC du SoC se fait sur un tty externe, on remarque également une différence de fréquence d'affichage, alors que sur la station de travail on a des affichages qui alternent Prod/Cons avec des décallages quelques fois, l'affichage sur le tty1 du simlateur systemC du SoC alterne des affichages par blocks de 4 prod / 4 cons, de manière plus synchrone, sans décallages.

\end{document}
