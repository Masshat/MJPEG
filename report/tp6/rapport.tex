\documentclass[12pt]{article}

\usepackage[utf8]{inputenc}
\usepackage[T1]{fontenc}
\usepackage[french]{babel}
\usepackage{float}
\usepackage[top=3cm, bottom=3cm, left=3cm, right=3cm]{geometry}
\usepackage{graphicx}
\usepackage{array}
\usepackage{setspace}

\floatplacement{figure}{H}
\newcommand{\HRule}{\rule{\linewidth}{0.5mm}}

\begin{document}
%\vspace{-1cm}
\begin{titlepage}
  \begin{center}
    \textsc{\LARGE Université Pierre et Marie Curie}\\[1.5cm]
    \includegraphics[height=1cm]{upmc.png}\\[1.5cm]
    \textsc{\Large Rapport Archi 4 }\\[2cm]
    \HRule \\[1cm]
    \textsc{\huge TP6 MJPEG }\\[0.5cm]
    \HRule \\[1cm]
    % Author and supervisor
    \noindent
    \begin{minipage}[t]{0.55\textwidth}
      \begin{flushleft} \large
        \emph{Auteurs:}\\
        BITAM \textsc{Massine}\\
        TOUMLILT \textsc{Ilyas}
      \end{flushleft}
    \end{minipage}%
    \begin{minipage}[t]{0.47\textwidth}
      \begin{flushright} \large
        \emph{Encadrant:} \\
        MEUNIER \textsc{quentin}
      \end{flushright}
    \end{minipage}
    \vfill
    % Bottom of the page
    {\large \today}
  \end{center}
\end{titlepage}
%\linespread{0.3}  % a suprimé si problème de compil
\section*{Question 1}
Nous avons opté pour une architcture clusterisé avec 4 processeurs par pipeline, 1 ram et un cache de 64 lignes de 8 mots par ligne.\\
Nous avons augmenter la tailles des canaux MWMR a la taille d'une demi image.
\section*{Question 2}
Cache : 32Ko *0,05*0,008= 1,608\\
RAM : 128 Ko *0,05 = 6,4\\
Mips 0,16 *8 = 1,28\\
MWMR =0,012*16=0,208\\
vgmn = 0,04\\
tg = 0,3\\
ramdac = 0,361\\
crossbar = 0,1\\
\\
	Total =10,297
\section*{Question 3}
La simulation a étant tres lente nous avons calculé que 8 images.\\
	notre fréquence est (85000000 *25 ) / 8 = 266 MHZ.\\
	consomation = 266 * 10,297 = 2739.
	
\section*{Question 4}
Ayant étant déçu par l'archi clustursé nous avons opté pour archi classqie avec 4 processeurs, mais cette fois ci avec un comprocesseurs idct, et des caches de 32 lignes.
Cache = ((32 * 8 * 4 * 2) * 4) = 8Ko * 0.05 + 0.004 = 0.44 \\
RAM = 43Ko * 0.05 = 2.15\\
Mips = 0.16 * 4 = 0.64\\
Mwmr = 0.012 * 16 = 0.208\\
tg = 0.3\\
ramdac = (((160*120*1.5) / 4) / 1024) * 0.05 + 0.01 = 0.361\\
\\
Total = 4.1mm2

notre fréquence est (tjours pour 8 images) = (70500000*25)/8 = 157 MHZ.
\section*{Question 6}
Notre meilleur Soc est sans doute le deuxieme, nous avons déduit que plus la vidéo étaits grande , il y'avait obligation d'utiliser des coprocesseurs matériel pour idct.

\end{document}
