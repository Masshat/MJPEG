\documentclass[12pt]{article}

\usepackage[utf8]{inputenc}
\usepackage[T1]{fontenc}
\usepackage[french]{babel}
\usepackage{float}
\usepackage[top=3cm, bottom=3cm, left=3cm, right=3cm]{geometry}
\usepackage{graphicx}
\usepackage{array}
\usepackage{setspace}

\floatplacement{figure}{H}
\newcommand{\HRule}{\rule{\linewidth}{0.5mm}}

\begin{document}
%\vspace{-1cm}
\begin{titlepage}
  \begin{center}
    \textsc{\LARGE Université Pierre et Marie Curie}\\[1.5cm]
    \includegraphics[height=1cm]{upmc.png}\\[1.5cm]
    \textsc{\Large Rapport Archi 4 }\\[2cm]
    \HRule \\[1cm]
    \textsc{\huge TP5 MJPEG }\\[0.5cm]
    \HRule \\[1cm]
    % Author and supervisor
    \noindent
    \begin{minipage}[t]{0.55\textwidth}
      \begin{flushleft} \large
        \emph{Auteurs:}\\
        BITAM \textsc{Massine}\\
        TOUMLILT \textsc{Ilyas}
      \end{flushleft}
    \end{minipage}%
    \begin{minipage}[t]{0.47\textwidth}
      \begin{flushright} \large
        \emph{Encadrant:} \\
        MEUNIER \textsc{quentin}
      \end{flushright}
    \end{minipage}
    \vfill
    % Bottom of the page
    {\large \today}
  \end{center}
\end{titlepage}
%\linespread{0.3}  % a suprimé si problème de compil
\section*{Question 1}
Il faudrait environ 28 millions de cycles pour la décompression.
\section*{Question 2}
Un coefficient est sur 4 octets, étant donné que la taille d'un pixel est de 1 octets, on peut donc transférer 4 pixels par cycle.\\
Le composant VciCrossbar est de 10. Pour une transaction ( aller + retour ), on a donc une latence de 20 cycles :\\
Prise de verrou (ll+sc) : 2x20cycles\\
Lecture de status: 20 cycles\\
Lecture/écriture de données: (64/8) x 20\\
Mise à jour du status: 20 cycles\\
Relachement du verrou: 20 cycles\\
\section*{Question 3}
Le gain apporté est beaucoup plus faible comparé au coût matériel.
\section*{Question 4}
Résultat des tests:
\begin{center}
\begin{tabular}{|c|c|}
\hline
Temps d'exec & Cycles (en millions)\\ \hline
100 & 26,7\\ \hline
500 & 26,6\\ \hline
1000 & 26,5\\ \hline
4000 & 26,4\\ \hline
8000 & 26,5\\ \hline
16000 & 27,1\\ \hline
\end{tabular}\newline
\end{center}
\section*{Question 5}
Tests avec le coprocesseur :
\begin{center}
\begin{tabular}{|c|c|}
\hline
Temps d'exec & Cycles (en millions)\\ \hline
160 & 26,5\\ \hline
576 & 26,4\\ \hline
704 & 26,5\\ \hline
1856 & 26,4\\ \hline
\end{tabular}\newline
\end{center}
En vue des résultats, il serait préférable d'utiliser le coprocesseur. Car même si les résultats en terme de performance sont similaires, le coût matériel est plus faible.

\section*{Question 6}
On pourrait, avec coût nul, tester une implémentation plus rapidement. Le coprocesseur virtuelle permet d'atteindre son objectif lorsqu'on veut se lancer dans la composition architecturale.

\end{document}
